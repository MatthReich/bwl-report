\chapter{Interne Zinsfußmethode}
\label{Interne Zinsfussmethode}
\section{Definition}
Die interne Zinsfußmethode gehört, wie die Kapitalwertmethode, zu einem der dynamischen Verfahren der Wirtschaftlichkeitsrechnung. Das Ziel dieses Verfahren ist es, den internen Zinsfuß auszurechnen, der oft auch als interne Rendite oder als interner Zinssatz bezeichnet wird. Der interne Zinsfuß stellt die Effektivverzinsung einer Investition dar, bei dem der Kapitalwert gleich null ist. Wie bei der Kapitalwertmethode geht man bei diesem Verfahren auch von einem vollständigen Markt aus. Der interne Zinsfuß dient, wie der Kapitalwert als eine weitere Entscheidungshilfe um korrekte Investition zu tätigen.\footnote{\cite{studyflix-interner-zinsfuss}}
\section{Formel}
Die interne Zinsfußmethode basiert auf der Kapitalwertmethode, weswegen wir wieder unsere Kapitalwertfunktion brauchen. Wie schon vorher erwähnt, ist der interne Zinssatz als Diskontierungszinssatz
definiert, bei dem der Barwert der Rückflüsse einer Investition zuzüglich des Barwertes des Liquidationserlöses gleich dem Barwert der Investitionsausgaben ist oder ganz einfach formuliert, bei dem der Kapitalwert gleich null ist.\\
Schlussfolgernd bedeutet, dies das wir die Nullstelle unserer Kapitalwertfunktion feststellen, indem wir folgende Gleichung nach $r$ auflösen:\footnote{\cite{lex-interner-zinsfuss}}\newpage
\begin{align*}
    KW = -Z_{0} + \sum \limits_{t=1}^{T}{\frac{Z_{t}}{(1+r)^{t}}} = 0
\end{align*}
\\
$Z_0$ = Die Anfangszahlung \\
$T$  = Die Betrachtungsdauer\\
$Z_t$ = Der Zahlungsstrom der Periode t. Besteht aus Einzahlungen - Auszahlungen.\\
$r$ = Kalkulationszinssatz\\
$t$ = Periode\\
$KW$ = Kapitalwert
\section{Berechnung vom internen Zinsfuß}
Wie man an der Gleichung erkennen kann bewirkt eine Investitionsdauer von n Jahren eine Kapitalwertfunktion n-ten Grades. Dies hat zur Folge, dass die Berechnung des internen Zinsfußes sich sehr erschwert. Eine Möglichkeit eine Funktion n-ten Grades zu lösen ist die lineare Interpolation, also dem Ausprobieren von verschiedenen Werten. Es gibt jedoch mehrere spezielle Fälle, die uns erlauben den internen Zinsfuß auch, ohne Hilfe von Interpolation zu berechnen.
\subsection{Zahlungsdauer von maximal 2 Jahren}
Der erste Spezialfall ist, wenn die Investition nur bis zu 2 Perioden umfasst. Dadurch wird die Formel zu einer quadratischen Gleichung die man auf gewohnter Weise lösen kann. Dazu ein kleines Beispiel mit $T = 2$, $Z_0 = -1000$ Euro, $Z_1 = 600$ Euro, $Z_2 = 600$ Euro,:
\begin{align*}                                                                               \\
    -1000 + \frac{600}{(1+r)^1} + \frac{600}{(1+r)^2} = 0 & \text{\phantom m}|:(-1000)                    \\
    1 - \frac{0,6}{1+r} - \frac{0,6}{(1+r)^2} = 0         & \text{\phantom m}|\cdot(1+r^2)                \\
    (1+r)^2 - 0,6\cdot(1+r) - 0,6 = 0                     & \text{\phantom m}|(1+r) = x                   \\
    x^2-0,6x-0,6=0                                        & \text{\phantom m}|\text{PQ oder Mitternachts} \\
    x_1 = 1,131 \text{\phantom p} x_2 = -0,531            & \text{\phantom m}|(1+r) = x                   \\
    r = 1,131 - 1                                                                                         \\
    r = 13,1 \text{ \%}
\end{align*}
\begin{center}
    Der interne Zinsfuß beträgt 13,1 \%.
\end{center}
\newpage
\subsubsection{Erklärung}
Wir fangen damit an, die Formel mit unseren Angaben auszufüllen. Danach formen wir die Gleichung so um, damit wir die $(1+r)$ in $x$ substituieren können. Danach setzen wir die PQ-
oder Mitternachtsformel ein, um die beiden $x$ auszurechnen und nehmen von den beiden Zahlen die positive. Jetzt substituieren wir wieder das $x$ zurück in $(1+r)$ und können so das $r$ berechnen, um unsere Rechnung zu beenden.
\subsection{Jährlich konstante Rückflüsse}
Der zweite Spezialfall ist einfacher als der erste. Man kann nähmlich genau sehen was der interne Zinsfuß beträgt. Ein kleines Beispiel dazu:
\begin{align*}
    KW = 0                   \\
    Z_0 = -3000 \text{ Euro} \\
    Z_1 = 150 \text{ Euro}   \\
    Z_2 = 150 \text{ Euro}   \\
    Z_3 = 150 \text{ Euro}   \\
    Z_4 = 150 \text{ Euro}   \\
    Z_5 = 3150 \text{ Euro}
\end{align*}
\begin{center}
    interne Zinsfuß beträgt 5 \%.
\end{center}
Man kann an den verschiedenen Cashflows sehen, dass wir in jeder Periode fünf Prozent unserer Anfangsauszahlung bekommen, Zusätzlich erhalten wir in der letzten Periode unsere Anfangsauszahlung plus die fünf Prozent zurück, daraus schlussfolgern wir, dass der interne Zinsfuß 5 Prozent betragen muss.
\subsection{Der Zweizahlungsfall}
Dieser spezial Fall macht sich dadurch bemerkbar, dass es neben der Anfangsinvestition nur noch zu einer weiteren Einzahlung zum Ende der Investition kommt. Es kommt zu keinen zwischenzeitlichen Ein- und Auszahlungen. Daraus resultiert folgende Kapitalwertgleichung, die man leicht umformen kann nach $r$:\footnote{\cite{lex-interner-zinsfuss}}\\
\begin{center}
    $ KW = -Z_{0} + \frac{Z_{t}}{(1+r)^{t}} \rightarrow r = \sqrt[t]{\frac{Z_t}{Z_0}} $
\end{center}
\newpage
\subsubsection{Beispiel}
Folgender Zerobound ist im Moment für 30 Euro erhältlich und soll in 20 Jahren für einen Wert von 250 Euro zurückbezahlt werden. Der interne Zinsfuß lässt sich wie folgend ausrechnen:
\begin{align*}
    r = \sqrt[20]{\frac{250}{30}} -1 = 0,11
\end{align*}
\begin{center}
    Der interne Zinsfuß beträgt in diesem Beispiel 11\%.
\end{center}
\section{Interpretation}
Um den internen Zinsfuß richtig interpretieren zu können muss man zunächst verstehen, dass es sich hier um eine Renditekennziffer handelt und nicht einem absoluten Maßstab wie beim Kapitalwert. Der interne Zinsfuß sollte außerdem auch nie, als alleiniges Kriterium verwendet werden, um eine Investitionsendscheidung zu treffen. Er sollte immer mit der Mindestrendite verglichen werden. Ein Investitionsobjekt ist erst dann Vorteilhaft, wenn der interne Zinsfuß mindestens so hoch ist wie der Kalkulationszinssatz. Am besten wird diese Regel mit dem Kapitalwert erklärt. Erzielt eine Investition einen positiven Kapitalwert entsteht neben Rückzahlung und Verzinsung des Kapitals ein Überschuss in Höhe des Kapitalwertes. Ist der interne Zinsfuß gleich des Kalkulationszinssatzes bedeutet, dass die Investition noch in der Lage ist die Investitionsausgaben zu amortisieren. Ist der interne Zinsfuß kleiner, als der Kalkulationszinssatz führt dies zu einem negativen Kapitalwert. Grundlegend sollten auch Investitionen durchgeführt werden, die einen höheren internen Zinsfuß als der Marktzinssatz haben. Da dies bedeutet, dass wir einen höheren Zinssatz als der Durchschnitt bekommen. Der interne Zinssatz ist nicht so einfach zu bewerten, weswegen man in der Praxis den Kapitalwert bevorzugt.
\footnote{\cite{lex-interner-zinsfuss}}\footnote{\cite{studyflix-interner-zinsfuss}}