\chapter{Annuitätenmethode}
\label{Annuitaetenmethode}

\section{Definition}


Mit der Annuitätenmethode wird der Nettobarwert in gleich hohe Mehrerträge pro Periode umgerechnet. Einen solchen Mehrertrag nennt man auch Annuität. Sofern die Laufzeit sowie die zu betrachtenden Investitionen gleich sind, liefert die Annuitätenmethode keine anderen Ergebnisse als die \hyperref[Kapitalwertmethode]{Kapitalwertmethode} oder \hyperref[Vermoegensendwertmethode]{Vermögensendwertmethode}. Jedoch ist die Annuitätenmethode nützlich, um die Investition auf andere Art und Weise zu analysieren. Zum Beispiel erlaubt die Annuitätenmethode, ohne die Berücksichtigung einer Differenzinvestition, Investition unterschiedlicher Anschaffungswerte und Nutzungsdauern vergleichen zu können.\footnote{\cite{bwllexicon-annu}} \footnote{\cite{wikipedia-annu}}


\section{Berechnung}

Die Annuität $a$ lässt sich nur errechnen, sofern der \ac{nbw} und der \linebreak \ac{kwf} bekannt sind. Der \ac{kwf} wird auch \ac{anf} genannt. Einfach gesagt ist die Annuität das Produkt aus dem \ac{nbw} und des \ac{anf}, dargestellt in der Formel \eqref{eq:annuitaeat}. Der \ac{anf} lässt sich durch den risikolosen Zinssatz am Kapitalmarkt $r$ und der Laufzeit der Investitionsprojektes $n$ errechnen. Dafür wird die Formel \eqref{eq:annuitaeatsfaktor} benutzt.\footnote{\cite{studyflix-annu}} \footnote{\cite{bwllexicon-annu}}

\begin{equation}
    a = NBW \cdot ANF
    \label{eq:annuitaeat}
\end{equation}

\begin{equation}
    ANF = \frac{ (1 + r)^n \cdot r }{ (1 + r)^n - 1 }
    \label{eq:annuitaeatsfaktor}
\end{equation}

\section{Beispielrechnung}

Herr Mustermann hat ein Investitionsprojekt mit einem Kapitalwert von 5000 € geplant. Diese möchte er sich über die nächsten 5 Jahre auszahlen lassen. Nun möchte Herr Mustermann wissen, über welchen Betrag er jährlich verfügen kann, wenn er einen Zinssatz von 8\% hat.


\subsection{Rechnung}

Um den Annuitätenfaktor zu errechnen, setzen wir in der dafür vorgesehen Formel \eqref{eq:annuitaeatsfaktor} n = 5 und r = 0,08.

\bigskip

ANF = $\frac{ (1 + 0,08)^5 \cdot 0,08 }{ (1 + 0,08)^5 - 1 }$ = 0,2504564545

\bigskip

\noindent
Um nun die Annuität zu bekommen, multipliziert man nach der Formel \eqref{eq:annuitaeat} die 5000 € mit dem zuvor ausgerechneten \ac{anf}:

\bigskip

Annuität = 5000 € $\cdot$ 0,2504564545 = 1252,2822725 € gekürzt 1252,28 €

\subsection{Interpretation}

Herr Mustermann hat eine Annuität von 1252,28 €, was bedeutet, dass ihm Jährlich 1252,28 € als gleichbleibende Summe zur Verfügung stehen.

\section{Bewertung}

Vorteil der Annuitätenmethode ist, dass die Zahlungen differenziert erfasst werden, weshalb eine einfache Vergleichbarkeit gegeben ist. Jedoch ist die Zuordnung der Zahlungen zu den Investitionsgütern schwierig, weil meistens mehrere Anlagen am Prozess beteiligt sind. Zudem beeinträchtigen äußerliche Einflüsse die Abschätzung von zukünftigen Zahlungen, was zu einer verfälschten Annuität führen kann.
