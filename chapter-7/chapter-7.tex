\chapter{Annuitätenmethode}
\label{Annuitaetenmethode}

\section{Definition}


Mit der Annuitätenmethode wird der Nettobarwert in gleich hohe Mehrerträge pro Periode umgerechnet. Einen solchen Mehrertrag nennt man auch Annuität. Sofern die Laufzeit sowie die zu betrachtenden Investitionen gleich sind, liefert die Annuitätenmethode keine anderen Ergebnisse als die \hyperref[Kapitalwertmethode]{Kapitalwertmethode} oder \hyperref[Vermoegensendwertmethode]{Vermögensendwertmethode}. Jedoch ist die Annuitätenmethode nützlich, um die Investition auf andere Art und Weise zu analysieren. Zum Beispiel erlaubt die Annuitätenmethode, ohne die Berücksichtigung einer Differenzinvestition, Investition unterschiedlicher Anschaffungswerte und Nutzungsdauern vergleichen zu können. Die Annuitätenmethode wird oft auch als Verrentung bezeichnet.\footnote{\cite{bwllexicon-annu}} \footnote{\cite{wikipedia-annu}}


\section{Berechnung}

Es gibt zwei mögliche Szenarien, in welchen man die Annuität errechnen möchte. Zum einen eine heutige Zahlung, bei welchem der aktuelle \ac{nbw} bekannt ist und auf die jeweiligen Jahre aufgezinst wird und zum anderen eine spätere Zahlung, bei der der Endwert des \ac{nbw} abgezinst wird.

\subsection{Verrentung einer heutigen Zahlung}

Um die Annuität $a$ einer heutigen Zahlung zu errechnen, benötigt man den \ac{nbw}$_0$ sowie den \ac{kwf}. Der \ac{kwf} wird auch als \ac{anf} bezeichnet. Einfach gesagt ist die Annuität das Produkt aus dem \ac{nbw} und des \ac{anf}, dargestellt in der Formel \eqref{eq:annuitaeat}. Der \ac{anf} lässt sich durch den risikolosen Zinssatz am Kapitalmarkt $r$ und der Laufzeit der Investitionsprojektes $n$ errechnen. Dafür wird die Formel \eqref{eq:annuitaeatsfaktor} benutzt.\footnote{\cite{studyflix-annu}} \footnote{\cite{bwllexicon-annu}}

\begin{equation}
    a = NBW_0 \cdot ANF
    \label{eq:annuitaeat}
\end{equation}

\begin{equation}
    ANF = \frac{ (1 + r)^n \cdot r }{ (1 + r)^n - 1 }
    \label{eq:annuitaeatsfaktor}
\end{equation}

\subsection{Verrentung einer späteren Zahlung}

Die Annuität $a$ einer späteren Zahlung benötigt einen angenommen Endwert nach einer Zeitperiode $n$, welcher wie folgend in der Formel \eqref{eq:annuitaeatSpaeter} dargestellt wird \ac{nbw}$_n$. Anstelle des in der Verrentung einer heutigen Zahlung verwendeten \ac{anf} wird bei der späteren Zahlung ein \ac{rvf} verwendet. Dieser berechnet sich aus der Formel \eqref{eq:restwertverteilungsfaktor}, wobei $r$ für den risikolosen Zinssatz am Kapitalmarkt steht und $n$ für die Zeitperiode.\footnote{\cite{rechnungswesen-annu}}

\begin{equation}
    a = NBW_n \cdot RVF_n
    \label{eq:annuitaeatSpaeter}
\end{equation}

\begin{equation}
    RVF_n = \frac{ r }{ (1 + r)^n - 1 }
    \label{eq:restwertverteilungsfaktor}
\end{equation}


\section{Beispielrechnung einer heutigen Zahlung}

Herr Mustermann hat ein Investitionsprojekt mit einem Kapitalwert von 5000 € geplant. Diese möchte er sich über die nächsten 5 Jahre auszahlen lassen. Nun möchte Herr Mustermann wissen, über welchen Betrag er jährlich verfügen kann, wenn er einen Zinssatz von 8\% hat.


\subsection{Rechnung}

Um den Annuitätenfaktor zu errechnen, setzen wir in der dafür vorgesehen Formel \eqref{eq:annuitaeatsfaktor} n = 5 und r = 0,08.

\bigskip

$ANF = \frac{ (1 + 0,08)^5 \cdot 0,08 }{ (1 + 0,08)^5 - 1 } = 0,2504564545$

\bigskip

\noindent
Um nun die Annuität zu bekommen, multipliziert man nach der Formel \eqref{eq:annuitaeat} die 5000 € mit dem zuvor ausgerechneten \ac{anf}:

\bigskip

$a = 5000 \text{€} \cdot 0,2504564545 = 1252,2822725 \text{€}$ gekürzt $1252,28 \text{€}$

\subsection{Interpretation}

Herr Mustermann hat eine Annuität von 1252,28 €, was bedeutet, dass ihm Jährlich 1252,28 € als gleichbleibende Summe zur Verfügung stehen.


\section{Beispielrechnung einer späteren Zahlung}

Herr Mustermann studiert \ac{ain} und ist kurz davor seinen Bachelor abzuschließen. Nachdem er in der Vorlesung \ac{bwl} von der Annuitätenmethode gehört hat, dachte er sich, was für einen Gehalt er wohl fordern müsste, wenn er nach 10 Jahren zu den Millionären gehören möchte. Nach kurzer Suche hat Herr Mustermann eine passende Bank gefunden, bei der er seinen Gehalt anlegen möchte. Diese verspricht ihm einen Zinssatz von 3\% für diese 10 Jahre.

\subsection{Berechnung}

Um den \ac{rvf} mit der Formel \eqref{eq:restwertverteilungsfaktor} werden der Zinssatz von 0,03 und die Zeitperiode von 10 Jahren benötigt.

\bigskip
$RVF_{10} = \frac{ 0,03 }{ (1 + 0,03)^{10} - 1 } = 0,0872305066$

\bigskip
\noindent
Nun setzt man den \ac{rvf} in die Formel \eqref{eq:annuitaeatSpaeter} für die spätere Zahlung ein, mit dem von Herrn Muster gewünschten Betrag von 1000000 €, um die Annuität zu berechnen:

\bigskip
$a = 1000000 \text{€} \cdot 0,0872305066 = 87230,5066 \text{€}$ gerundet $87230,51 \text{€}$

\subsection{Interpretation}

Wenn Herr Muster nach 10 Jahren Millionär sein möchte, muss er einen vollen Jahresgehalt von 87230,51 € in die Bank einzahlen. Das wäre ein Monatsgehalt von 7269,21€.

\section{Bewertung}

Vorteil der Annuitätenmethode ist, dass die Zahlungen differenziert erfasst werden, weshalb eine einfache Vergleichbarkeit gegeben ist. Jedoch ist die Zuordnung der Zahlungen zu den Investitionsgütern schwierig, weil meistens mehrere Anlagen am Prozess beteiligt sind. Zudem beeinträchtigen äußerliche Einflüsse die Abschätzung von zukünftigen Zahlungen, was zu einer verfälschten Annuität führen kann.
