\chapter{Kapitalwertmethode}
\label{Kapitalwertmethode}

\section{Definition}
Die Kapitalwertmethode ist ein Verfahren der dynamischen Investitionsrechnung. Mit dieser Methode wird der Kapitalwert ausgerechnet, der die Summe aller Einzahlungen und Auszahlungen, auf den heutigen Stand abzinst und darstellt. Dieser Kapitalwert bildet sehr oft die Grundlage für Investitionsendscheidungen.

\section{Formel}
Die Kapitalwertmethode wird mit folgender Formel ausgerechnet:
\begin{align*}
    KW = -Z_{0} + \sum \limits_{t=1}^{T}{\frac{Z_{t}}{(1+r)^{t}}}
\end{align*}
\\
$Z_0$ = Die Anfangszahlung \\
$T$  = Die Betrachtungsdauer\\
$Z_t$ = Der Zahlungsstrom der Periode t. Besteht aus Einzahlungen - Auszahlungen.\\
$r$ = Kalkulationszinssatz\\
$t$ = Periode\\ \\
\newpage
\section{Interpretation}
Um mit der Kapitalwertmethoden eine Investitionsendscheidung zu treffen, sollte man ihn wie folgend interpretieren:
\begin{itemize}
    \item Wenn der Kapitalwert gleich 0 ist, bedeutet das für die Investition, dass wir unser eingesetztes Kapital auch wieder zurückbekommen. Es kommt bei so welchen Investitionen zu keinem Vorteil und Nachteil bzw. zu keinem Gewinn oder Verlust.
    \item Wenn der Kapitalwert größer als 0 ist, bedeutet das für die Investition, dass sie einen Gewinn einbringen wird und es empfehlenswert ist diese Investition durchzuführen.
    \item Wenn der Kapitalwert kleiner als 0 ist, bedeutet das für die Investition, dass sie Verluste einbringen wird. Solche Investitionen sollte man vermeiden.
\end{itemize}

\section{Beispiel}
Herr Mustermann möchte mit den steigenden Immobilienpreisen profitieren. Dafür überlegt er sich eine Immobilie für 300.000 Euro zu erwerben. Nach 2 Jahren würde er diese Immobilie wieder mit Gewinn für 320.000 Euro verkaufen. Als Alternative kann Herr Mustermann allerdings bei der Bank für den gleichen Zeitraum risikoarm seine Investition als Festgeld anlegen mit einem Zinssatz von 3 Prozent.\\ \\
Um seine Investitionsendscheidung zu treffen, rechnet Herr Mustermann den Kapitalwert wie folgt aus:\\

\begin{align*}
    320000 \div 1.03^2 \approx 301630 \\
    KW = 301630 - 300000 = 1630
\end{align*}
\\
Aus der Rechnung heraus zeigt sich, dass der Kapitalwert positiv ist und sich die Investition in die Immobilie empfehlenswert ist. Jedoch sollte man trotzdem beachten, dass bei der Immobilie ein größeres Risiko besteht, als wie bei der Bank.

\footnote{https://studyflix.de/wirtschaft/kapitalwertmethode-71 : 27 Dez 2021}
\footnote{https://welt-der-bwl.de/Kapitalwertmethode: 27 Dez 2021 : Beispiel etwas abgeändert}
