\chapter{Vermögensendwertmethode}
\label{Vermoegensendwertmethode}

\section{Definition}
Unter der Vermögensendwertmethode versteht man ein dynamisches Investitionsverfahren, bei dem eine durch eine Investition ausgelöste Zahlungsreihe auf einen späteren Zeitpunkt aufgezinst wird.\footnote{\cite{finanzen-vermoegensendwertmethode}} Diese Methode ist ähnlich wie die  \hyperref[Kapitalwertmethode]{Kapitalwertmethode}, bezieht sich aber auf den Endwert als Entscheidungsgrundlage.\\
Die Methode wird auf verschiedene Investitionen angewandt. Danach können die jeweiligen Endwerte miteinander verglichen werden, um das Provitabelste auszuwählen. Ein negativer Wert deutet auf eine eher schlechte Investition hin.\footnote{\cite{bwllernen-endwertmethode}}

\section{Berechnung}
Um den Vermögensendwert zu errechnen, wird folgende Formel \eqref{eq:Vermoegensendwert}\footnote{\cite{studyflix-endwertmethode}} verwendet:

\begin{equation}
    V_T = \sum_{ t = 0 }^{ T }{(E_{ t } - A_{ t })(1 + r)^{T - t}}
    \label{eq:Vermoegensendwert}
\end{equation}
\smallskip\\
Hierbei steht E zum Zeitpunkt t für die Einzahlung und A zum Zeitpunkt t für die Auszahlung. Die Differenz wird mit dem Zinssatz r multipliziert. Der Zinssatz ist zudem abhängig von dem Zeitpunkt, da sich dieser über jede Periode mitzieht. Der Vermögensendwert berechent sich demzufolge aus der Summe der Differenz der Ein- und Ausgaben, auf welche der vom Jahr abhängige Zinssatz multipliziert wurde.\footnote{\cite{studyflix-endwertmethode}}\\

Anzumerken ist, dass hier nicht mit gänzlich realen Werten gerechnet wird. Der Zinssatz wird vorher kalkuliert und auf geschätze zukünfitige Zahlungen angewandt.\footnote{\cite{finanzen-vermoegensendwertmethode}}

\section{Beispielrechnung}

In der Tabelle \ref{tb:VermoegensendwertRechnung} wird exemplarisch eine Berechnung mit der Endwertmethode dargestellt.
\bigskip

\begin{table}[!h]
    \caption{Beispiel einer Vermögensrechnung auf 5 Jahre mit einem Zinssatz von 10\%}
    \begin{tabular}{llllll}
        \cline{1-6} \rowcolor{gray}
        Jahre            & 1                              & 2                              & 3                             & 4          & 5      \\ \cline{1-6} \rowcolor{white}
        Anschaffung      & -5000                          &                                &                               &            &        \\ \cline{1-6} \rowcolor{white}
        Einzahlung       &                                & 1000                           & 3000                          & 5000       & 15000  \\ \cline{1-6} \rowcolor{white}
        Auszahlung       &                                & -2000                          & -1500                         & -3000      & -5000  \\ \cline{1-6} \rowcolor{white}
                         & -5000 * 1,1\textsuperscript{4} & -1000 * 1,1\textsuperscript{3} & 1500 * 1,1\textsuperscript{2} & 2000 * 1,1 & 10000  \\ \rowcolor{white}
        Vermögensendwert &                                &                                &                               &            & 5363,5 \\ \cline{1-6}
    \end{tabular}
    \label{tb:VermoegensendwertRechnung}
\end{table}

\bigskip

\noindent
Im ersten Jahr wurde nur eine Anschaffung von einem Betrag von 5000 getätigt. Da noch vier Jahre bis zu dem gewünschten Vermögensendwert sind, wird der Zinsatz mit vier exponiert. Als erstes Zwischenergebnis hat man im ersten Jahr einen Wert von -7320,5. In den folgenden drei Jahren wurden jeweils Ein- sowie Auszahlungen getätigt, wobei deren Differenz mit dem Zinssatz, auch abhängig von der Dauer zu dem gewünschten Jahr, multipliziert wurde. Dementsprechend erhält man die Werte, von Jahr zwei ausgehend, -1331, 1815 und 2200. Im dem letzen zu berechnenden Jahr, also dem, von welchem der Vermögensendwert berechnet wird, wird der Zinssatz vernachlässigt, da dieser keine Rolle mehr spielt. Dementsprechend wird nur die Differenz, 10000, berechnet. Die Summe der jeweiligen Zwischenergebnisse ergbit dann den Vermögensendwert von 5363,5. \linebreak
Da dieser Wert positiv ist, ist dies eine Investition, die man tätigen könnte.
