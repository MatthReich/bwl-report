\chapter{Vermögensendwertmethode}
\label{Vermoegensendwertmethode}

\section{Definition}

Unter der Vermögensendwertmethode versteht man ein dynamisches Investitionsverfahren, bei welchem eine durch eine Investition ausgelöste Zahlungsreihe auf einen späteren Zeitpunkt aufgezinst wird.\footnote{\cite{finanzen-vermoegensendwertmethode}} Diese Methode ist ähnlich zu der \hyperref[Kapitalwertmethode]{Kapitalwertmethode}, bezieht sich aber auf den Endwert als Entscheidungsgrundlage.\\
Wird die Methode auf verschiedene Investitionen angewandt, können die jeweiligen Endwerte miteinander verglichen werden, um das Profitabelste auszuwählen. Ein negativer Wert deutet auf eine fragwürdige Investition hin.\footnote{\cite{bwllernen-endwertmethode}}

\section{Berechnung}

Um den Vermögensendwert zu errechnen, wird folgende Formel \eqref{eq:Vermoegensendwert}\footnote{\cite{studyflix-endwertmethode}} verwendet:

\begin{equation}
    V_T = \sum_{ t = 0 }^{ T }{(E_{ t } - A_{ t })(1 + r)^{T - t}}
    \label{eq:Vermoegensendwert}
\end{equation}

\bigskip

\noindent
Hierbei steht $E$ zum Zeitpunkt $t$ für die Einzahlung und $A$ zum Zeitpunkt $t$ für die Auszahlung. Die Differenz wird mit dem Zinssatz $r$ multipliziert. Der Zinssatz ist zudem abhängig von dem Zeitpunkt, da die Zahlung in der Folgeperiode erneut mit der bereits verzinsten Zahlung verzinst wird. Der Vermögensendwert berechnet sich demzufolge aus der Summe der Differenz der Ein- und Ausgaben, auf welche der vom Jahr abhängige Zinssatz multipliziert wurde.\footnote{\cite{studyflix-endwertmethode}}\\

\noindent
Anzumerken ist, dass hier nicht mit gänzlich realen Werten gerechnet wird. Der Zinssatz wird vorher kalkuliert und auf geschätzte zukünftige Zahlungen angewandt.\footnote{\cite{finanzen-vermoegensendwertmethode}}

\section{Beispielrechnung 1}

Das Unternehmen Muster tätigt eine Anschaffung von 5000 €. In den folgenden Jahren tätigt die Firma mehrere Ein- sowie Auszahlungen, welche in der Tabelle \ref{tb:VermoegensendwertDaten} dargestellt sind. Die Werte in der Tabelle sind als Betrag in Euro anzusehen.

\bigskip

\begin{table}[!h]
    \begin{tabular}{llllll}
        \cline{1-6} \rowcolor{gray}
        Jahre       & 1     & 2     & 3     & 4     & 5     \\ \cline{1-6} \rowcolor{white}
        Anschaffung & -5000 &       &       &       &       \\ \cline{1-6} \rowcolor{white}
        Einzahlung  &       & 1000  & 3000  & 5000  & 15000 \\ \cline{1-6} \rowcolor{white}
        Auszahlung  &       & -2000 & -1500 & -3000 & -5000 \\ \cline{1-6} \rowcolor{white}
    \end{tabular}
    \caption{Beispielzahlungen}
    \label{tb:VermoegensendwertDaten}
\end{table}

\bigskip
\noindent
Nun möchte Unternehmen Muster den Vermögensendwert der oben dargestellten Zahlungsreihe \eqref{tb:VermoegensendwertDaten}, welche sich über 5 Jahre streckt, errechnen. Dabei wird von einem Zinssatz von 10\% ausgegangen.
\subsection{Rechnung}

Im ersten Jahr wurde eine Anschaffung von 5000 € getätigt. Da noch vier Jahre bis zu dem betrachteten Endzeitpunkt fehlen, wird der Zinssatz mit vier exponiert. Als erstes Zwischenergebnis erhält man im ersten Jahr einen Wert von $-5000 \cdot 1,1^4 =$ -7320,5 €. In den folgenden drei Jahren wurden jeweils Ein- sowie Auszahlungen getätigt, wobei deren Differenz mit dem Zinssatz, ebenfalls abhängig von der Restlaufzeit der betrachteten Zeitspanne, multipliziert wurde. Dementsprechend erhält man die Werte, von Jahr zwei ausgehend, -1331 €, 1815 € und 2200 €. In dem letzten zu berechnenden Jahr der Zahlungsreihe, also dem Jahr des gesuchten Vermögensendwertes, wird der Zinssatz vernachlässigt, da von einer Zahlung am Ende des Jahres ausgegangen wird. Dementsprechend wird nur die Differenz, 10000 €, berücksichtigt. Die Summe der jeweiligen Zwischenergebnisse ergibt dann einen Vermögensendwert von 5363,5 €. Diese Berechnung wird auch in folgender Tabelle \ref{tb:VermoegensendwertRechnung} dargestellt. Die dort verwendeten Werte sind als Betrag in Euro anzusehen.

\begin{table}[!h]
    \begin{tabular}{llllll}
        \cline{1-6} \rowcolor{gray}
        Jahre            & 1                                    & 2                                    & 3                                   & 4                & 5      \\ \cline{1-6} \rowcolor{white}
        Anschaffung      & -5000                                &                                      &                                     &                  &        \\ \cline{1-6} \rowcolor{white}
        Einzahlung       &                                      & 1000                                 & 3000                                & 5000             & 15000  \\ \cline{1-6} \rowcolor{white}
        Auszahlung       &                                      & -2000                                & -1500                               & -3000            & -5000  \\ \cline{1-6} \rowcolor{white}
                         & -5000 $\cdot$ 1,1\textsuperscript{4} & -1000 $\cdot$ 1,1\textsuperscript{3} & 1500 $\cdot$ 1,1\textsuperscript{2} & 2000 $\cdot$ 1,1 & 10000  \\ \rowcolor{white}
        Vermögensendwert &                                      &                                      &                                     &                  & 5363,5 \\ \cline{1-6}
    \end{tabular}
    \caption{Beispiel einer Vermögensrechnung auf 5 Jahre mit einem Zinssatz von 10\%}
    \label{tb:VermoegensendwertRechnung}
\end{table}

\subsection{Interpretation}

Ohne Betrachtung von anderen Vermögensendwerten ist das Ergebnis von 5363,5 € ein Wert, bei dem man durchaus eine Investition in Betracht ziehen könnte. Die positive Zahl deutet darauf hin, dass kein Verlust entsteht. Ein Verlust wäre an einem negativen Ergebnis erkennbar.

\section{Beispielrechnung 2}\label{beispielrechnungVer}

Der \ac{ain} Student Herr Muster arbeitet neben seinem Studium als Werkstudent in einer Firma. Von seinem dort verdienten Geld möchte er 500 € anlegen und in 5 Jahren mit dem Geld in den Urlaub fahren. Bei seiner suche nach einer passenden Bank sind ihm zwei besonders aufgefallen. Zum einen die Bank Euromacher und zum anderen die Anlegerbank. Der Zinssatz der Bank Euromacher beläuft sich auf 13\%, wohingegen die Anlegerbank einen Zinssatz von 10\% hat. In der Tabelle \eqref{tb:VermoegensendwertBanken} sind die jeweiligen Auszahlen der Banken dargestellt. Die Beträge sind in Euro zu betrachten. \pagebreak

\begin{table}[!h]
    \begin{tabular}{llllll}
        \cline{1-6} \rowcolor{gray}
        Jahre       & 1    & 2   & 3   & 4   & 5   \\ \cline{1-6} \rowcolor{white}
        Euromacher  & -500 & 100 & 100 & 250 & 350 \\ \cline{1-6} \rowcolor{white}
        Anlegerbank & -500 & 200 & 200 & 100 & 300 \\ \cline{1-6} \rowcolor{white}
    \end{tabular}
    \caption{Die Angebote der Banken für Herr Mustermann}
    \label{tb:VermoegensendwertBanken}
\end{table}

\noindent
Herr Mustermann wundert sich, wieso er bei beiden Banken in der Summe den gleichen Betrag bekommt, aber der Zinssatz der Bank Euromacher höher ist. Für ihn ist es eigentlich schon klar, welche Bank er bevorzugt. Jedoch möchte er gerne wissen, ob das wirklich stimmt.

\subsection{Berechnung}

Zuerst wird der Vermögensendwert der Bank Euromacher nach der Formel \eqref{eq:Vermoegensendwert} berechnet.

\bigskip
$V_{5, Euromacher} =$\\
\hspace*{10mm}$-500 \text{€} \cdot 1,13^4 + 100 \text{€} \cdot 1,13^3 + 100 \text{€} \cdot 1,13^2 + 250 \text{€} \cdot 1,13^1 + 350 \text{€} = 89,2428950000 \text{€}$\\
\hspace*{10mm}gekürzt $89,24 \text{€}$

\bigskip
\noindent
Nach der gleichen Methode wird auch der Vermögensendwert der Anlegerbank berechnet.

\bigskip
$V_{5, Anlegerbank} =$\\
\hspace*{10mm}$-500 \text{€} \cdot 1,10^4 + 200 \text{€} \cdot 1,10^3 + 200 \text{€} \cdot 1,10^2 + 100 \text{€} \cdot 1,10^1 + 300 \text{€} = 186,15 \text{€}$

\bigskip
\bigskip
\noindent
In folgender Tabelle \eqref{tb:VermoegensendwertBankenEnd} werden die Ergebnisse nochmals visuell besser dargestellt. Alle Werte sind in Euro zu betrachten.

\bigskip

\begin{table}[!h]
    \begin{tabular}{lllllll}
        \cline{1-6} \rowcolor{gray}
                    & Jahr 1 & Jahr 2 & Jahr 3 & Jahr 4 & Jahr 5 & Vermögensendwert \\ \cline{1-7} \rowcolor{white}
        Euromacher  & -500   & 100    & 100    & 250    & 350    & 89,24            \\ \cline{1-7} \rowcolor{white}
        Anlegerbank & -500   & 200    & 200    & 100    & 300    & 186,15           \\ \cline{1-7} \rowcolor{white}
    \end{tabular}
    \caption{Die Vermögensendwerte der Banken}
    \label{tb:VermoegensendwertBankenEnd}
\end{table}

\subsection{Interpretation}

Obwohl die Bank Euromacher mit einem höheren Zinssatz wirbt, erhält Herr Mustermann am Ende 96,91 € mehr von der Anlegerbank. Demzufolge hätte er, ohne die Betrachtung mit der Vermögensendwertmethode, eine Investition getätigt die ihm ein Bonus von 89,24 € verspricht. Nun weiß er aber, dass sich die Anlegerbank deutlich mehr rentiert und hat nach den 5 Jahren 186,15 € mehr für seinen Urlaub.

\section{Bewertung}

Wie bereits aufgezeigt, bezieht die Vermögensendwertmethode Werte mit in die Berechnung ein, welche noch nicht realisiert wurden. Daher kann sie lediglich als Schätzung erachtet werden. Trotzdem bietet sie eine gute Einschätzung um zwischen mehreren Investitionsmöglichkeiten zu entscheiden, wie in der zweiten \hyperref[beispielrechnungVer]{Beispielrechnung} zu sehen ist.
