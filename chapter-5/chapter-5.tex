\chapter{Vermögensendwertmethode}
\label{Vermoegensendwertmethode}

\section{Definition}

Unter der Vermögensendwertmethode versteht man ein dynamisches Investitionsverfahren, bei welchem eine durch eine Investition ausgelöste Zahlungsreihe auf einen späteren Zeitpunkt aufgezinst wird.\footnote{\cite{finanzen-vermoegensendwertmethode}} Diese Methode ist ähnlich zu der \hyperref[Kapitalwertmethode]{Kapitalwertmethode}, bezieht sich aber auf den Endwert als Entscheidungsgrundlage.\\
Wird die Methode auf verschiedene Investitionen angewandt, können die jeweiligen Endwerte miteinander verglichen werden, um das Profitabelste auszuwählen. Ein negativer Wert deutet auf eine fragwürdige Investition hin.\footnote{\cite{bwllernen-endwertmethode}}

\section{Berechnung}

Um den Vermögensendwert zu errechnen, wird folgende Formel \eqref{eq:Vermoegensendwert}\footnote{\cite{studyflix-endwertmethode}} verwendet:

\begin{equation}
    V_T = \sum_{ t = 0 }^{ T }{(E_{ t } - A_{ t })(1 + r)^{T - t}}
    \label{eq:Vermoegensendwert}
\end{equation}

\bigskip

\noindent
Hierbei steht $E$ zum Zeitpunkt $t$ für die Einzahlung und $A$ zum Zeitpunkt $t$ für die Auszahlung. Die Differenz wird mit dem Zinssatz $r$ multipliziert. Der Zinssatz ist zudem abhängig von dem Zeitpunkt, da die Zahlung in der Folgeperiode erneut mit der bereits verzinsten Zahlung verzinst wird. Der Vermögensendwert berechnet sich demzufolge aus der Summe der Differenz der Ein- und Ausgaben, auf welche der vom Jahr abhängige Zinssatz multipliziert wurde.\footnote{\cite{studyflix-endwertmethode}}\\

\noindent
Anzumerken ist, dass hier nicht mit gänzlich realen Werten gerechnet wird. Der Zinssatz wird vorher kalkuliert und auf geschätzte zukünftige Zahlungen angewandt.\footnote{\cite{finanzen-vermoegensendwertmethode}}

\section{Beispielrechnung}

Das Unternehmen Muster tätigt eine Anschaffung von 5000 €. In den folgenden Jahren tätigt die Firma mehrere Ein- sowie Auszahlungen, welche in der Tabelle \ref{tb:VermoegensendwertDaten} dargestellt sind. Die Werte in der Tabelle sind als Betrag in Euro anzusehen.

\bigskip

\begin{table}[!h]
    \begin{tabular}{llllll}
        \cline{1-6} \rowcolor{gray}
        Jahre       & 1     & 2     & 3     & 4     & 5     \\ \cline{1-6} \rowcolor{white}
        Anschaffung & -5000 &       &       &       &       \\ \cline{1-6} \rowcolor{white}
        Einzahlung  &       & 1000  & 3000  & 5000  & 15000 \\ \cline{1-6} \rowcolor{white}
        Auszahlung  &       & -2000 & -1500 & -3000 & -5000 \\ \cline{1-6} \rowcolor{white}
    \end{tabular}
    \caption{Beispielzahlungen}
    \label{tb:VermoegensendwertDaten}
\end{table}

\bigskip
\noindent
Nun möchte Unternehmen Muster den Vermögensendwert der oben dargestellten Zahlungsreihe \eqref{tb:VermoegensendwertDaten}, welche sich über 5 Jahre streckt, errechnen. Dabei wird von einem Zinssatz von 10\% ausgegangen.
\subsection{Rechnung}

Im ersten Jahr wurde eine Anschaffung von 5000 € getätigt. Da noch vier Jahre bis zu dem betrachteten Endzeitpunkt fehlen, wird der Zinssatz mit vier exponiert. Als erstes Zwischenergebnis erhält man im ersten Jahr einen Wert von $-5000 \cdot 1,1^4 =$ -7320,5 €. In den folgenden drei Jahren wurden jeweils Ein- sowie Auszahlungen getätigt, wobei deren Differenz mit dem Zinssatz, ebenfalls abhängig von der Restlaufzeit der betrachteten Zeitspanne, multipliziert wurde. Dementsprechend erhält man die Werte, von Jahr zwei ausgehend, -1331 €, 1815 € und 2200 €. In dem letzten zu berechnenden Jahr der Zahlungsreihe, also dem Jahr des gesuchten Vermögensendwertes, wird der Zinssatz vernachlässigt, da von einer Zahlung am Ende des Jahres ausgegangen wird. Dementsprechend wird nur die Differenz, 10000 €, berücksichtigt. Die Summe der jeweiligen Zwischenergebnisse ergibt dann einen Vermögensendwert von 5363,5 €. Diese Berechnung wird auch in folgender Tabelle \ref{tb:VermoegensendwertRechnung} dargestellt. Die dort verwendeten Werte sind als Betrag in Euro anzusehen.

\begin{table}[!h]
    \begin{tabular}{llllll}
        \cline{1-6} \rowcolor{gray}
        Jahre            & 1                                    & 2                                    & 3                                   & 4                & 5      \\ \cline{1-6} \rowcolor{white}
        Anschaffung      & -5000                                &                                      &                                     &                  &        \\ \cline{1-6} \rowcolor{white}
        Einzahlung       &                                      & 1000                                 & 3000                                & 5000             & 15000  \\ \cline{1-6} \rowcolor{white}
        Auszahlung       &                                      & -2000                                & -1500                               & -3000            & -5000  \\ \cline{1-6} \rowcolor{white}
                         & -5000 $\cdot$ 1,1\textsuperscript{4} & -1000 $\cdot$ 1,1\textsuperscript{3} & 1500 $\cdot$ 1,1\textsuperscript{2} & 2000 $\cdot$ 1,1 & 10000  \\ \rowcolor{white}
        Vermögensendwert &                                      &                                      &                                     &                  & 5363,5 \\ \cline{1-6}
    \end{tabular}
    \caption{Beispiel einer Vermögensrechnung auf 5 Jahre mit einem Zinssatz von 10\%}
    \label{tb:VermoegensendwertRechnung}
\end{table}

\subsection{Interpretation}

Ohne Betrachtung von anderen Vermögensendwerten ist das Ergebnis von 5363,5 € ein Wert, bei dem man durchaus eine Investition in Betracht ziehen könnte. Die positive Zahl deutet darauf hin, dass kein Verlust entsteht. Ein Verlust wäre an einem negativen Ergebnis erkennbar.

\section{Bewertung}

Wie bereits aufgezeigt, bezieht die Vermögensendwertmethode Werte mit in die Berechnung ein, welche noch nicht realisiert wurden. Daher kann sie lediglich als Schätzung erachtet werden. Trotzdem bietet sie eine gute Einschätzung um zwischen mehreren Investitionsmöglichkeiten zu entscheiden.
