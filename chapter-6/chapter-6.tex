\chapter{Geschäftswertbeitrag}
\label{Geschaeftswertbeitrag}

\section{Definition}

Der \ac{gwb}, im Englischen \ac{eva}, ist eine absolute Wertbeitragskennzahl, welche, vereinfacht, der Differenz zwischen den Kapitalerlösen und den Kapitalkosten entspricht. Demnach wird genau dann zusätzlicher Wert geschaffen, wenn über die Kapitalkosten für Eigen- und Fremdkapital hinaus verdient wird. Der \ac{gwb} wurde in den 1990er Jahren in der Unternehmensberatung Stern Stewart entwickelt.\footnote{\cite{wikipedia-eva}} \footnote{\cite{controlling-eva}}

\section{Berechnung}

Der Geschäftswertbeitrag setzt sich aus drei Elementen zusammen: Dem operativen Gewinn nach Steuern (\ac{nopat}), dem betriebsnotwendigem Vermögen (\ac{noa}) und den gewichteten durchschnittlichen Kapitalkosten (\ac{wacc}).\\

\noindent
Der \ac{nopat} berechnet sich aus der Differenz von dem Betriebsergebnis und den Steuern. Auch wenn die Berechnung des \ac{nopat} nicht ganz genau ist, kann man diesen Wert als Gewinn betrachten. Zudem kann man ihn auch als Richtlinie für den operativen Erfolg eines Unternehmens betrachten.\footnote{\cite{studyflix-nopat}} Der \ac{wacc} wird auch als gewichtete Kapitalkosten bezeichnet. Diese berechnen sich, in dem Eigenkapital und Fremdkapital abzüglich möglicher Steuervorteile gegeneinander aufgewogen werden. Dadurch lässt sich der durchschnittliche Verzinsungsanspruch der Kapitalgeber auf ihr eingesetztes Kapital verdeutlichen. Je niedriger der \ac{wacc} umso höher der Wert des Unternehmens.\footnote{\cite{studyflix-wacc}} Die \ac{noa} beschreibt das Fremd- sowie Eigenkapital das zur Erreichung des Betriebszwecks notwendig ist und setzt sich dementsprechend nur aus den Bestandteilen zusammen, die dem Betriebszweck dienen.\footnote{\cite{unternehmerinfos-noa}}

\bigskip
\noindent
Es gibt zwei Methoden, um den \ac{gwb} zu berechnen. Den subtraktiven Ansatz und den multiplikativen Ansatz. Beide Ansätze führen zum gleichen Berechnungsergebnis. Sie unterscheiden sich letztlich nur in der Fokussierung auf das absolute oder relative Erfolgsziel.

\subsection{Subtraktiver Ansatz}

Bei dem subtraktiven Ansatz werden von dem operativen Jahresergebnis die durchschnittlichen Kapitalkosten mal dem betriebsnotwendigem Vermögen abgezogen. Folgende Formel \eqref{eq:subtraktiver-geschaeftswertbeitrag}\footnote{\cite{wikipedia-eva}} repräsentiert diese Rechnung:

\begin{equation}
    GWB = NOPAT - WACC \cdot NOA
    \label{eq:subtraktiver-geschaeftswertbeitrag}
\end{equation}

\subsection{Multiplikativer Ansatz}

Bei dem multiplikativen Ansatz werden von der (Ist-)Gesamtkapitalrendite (\ac{irr}) die durchschnittlichen Kapitalkosten abgezogen und auf dieses Ergebnis wird dann das betriebsnotwendige Vermögen multipliziert. Dies wird in folgender Formel \eqref{eq:mupliplikativ-gwb-geschaeftswertbeitrag}\footnote{\cite{wikipedia-eva}} dargestellt. In der Formel \eqref{eq:mupliplikativ-irr-geschaeftswertbeitrag}\footnote{\cite{controllingportal-eva}} wird die Berechnung der \ac{irr} für die Vollständigkeit dargestellt. Die \ac{irr} berechnet sich aus dem Quotienten des operativen Gewinns nach Steuern und dem betriebsnotwendigem Vermögen multipliziert mit 100.

\begin{equation}
    GWB = (IRR - WACC) \cdot NOA
    \label{eq:mupliplikativ-gwb-geschaeftswertbeitrag}
\end{equation}

\begin{equation}
    IRR = \frac{NOPAT}{NOA} \cdot 100
    \label{eq:mupliplikativ-irr-geschaeftswertbeitrag}
\end{equation}

\bigskip

\noindent
Voraussetzung für diese Methode ist, dass der \ac{nopat} immer größer als die Kapitalkosten ist, welche bei der Investition anfallen, ist.\footnote{\cite{bwllexicon-eva}}

\section{Beispielrechnung 1}

Da der Fokus auf der Berechnung des Geschäftswertbeitrags liegt, werden für die Beispielrechnungen die Werte bereits angenommen. Demzufolge beträgt der \ac{wacc} 8\%, der \ac{nopat} beträgt 10000 € und die \ac{noa} belaufen sich auf 90000 €.

\subsection{Rechnung}

\subsubsection{Subtraktiver Ansatz}

Mit der Anwendung des subtraktiven Ansatzes \eqref{eq:subtraktiver-geschaeftswertbeitrag} ergibt sich folgende Rechnung:

\bigskip
$GWB = 10000 \text{€} - 0,08 \cdot 90000 \text{€} = 2800 \text{€}$


\subsubsection{Multiplikativer Ansatz}

Da beide Ansätze dasselbe Ergebnis haben sollten, sollte auch der multiplikative Ansatz \eqref{eq:mupliplikativ-gwb-geschaeftswertbeitrag} einen \ac{gwb} von 2800 ergeben:

\bigskip
$GWB = ((\frac{10000 \text{€}}{90000 \text{€}}) - 0,08) \cdot 90000 \text{€} = 2800 \text{€}$

\subsection{Interpretation}

Der Geschäftswertbeitrag von 2800 € zeigt, dass die Rendite über den Kosten für das eingesetzte Kapital liegt, weshalb diese Investition durchaus durchführbar ist und Werte schaffen wird. Wäre der Wert negativ, würde die Investition Verluste aufweisen und es wäre davon abzuraten, diese zu tätigen.

\section{Beispielrechnung 2}

Das Unternehmen Muster wird für zwei Quartale betrachtet. In dem ersten Quartal hat das Unternehmen 100000 € in ein Projekt investiert, welches sie in Zukunft voran bringen soll. Der \ac{wacc} beläuft sich auf 8\% und der \ac{nopat} beläuft sich auf 10000 €. Im zweiten Quartal investiert das Unternehmen Muster mit gleichbleibendem \ac{wacc} aber um 5000€ erhöhtem \ac{nopat} einen Wert von 200000 €. Nun möchte das Unternehmen wissen, ob sie in den zwei Quartalen ihren Unternehmenswert erhöhen konnten, oder ob dieser gesunken ist. Falls dieser gesunken ist, möchte das Unternehmen gerne wissen, was für Möglichkeiten es hat, ihren Wert wieder zu steigern.\\
Um die angegeben Daten nochmal visuell besser dargestellt zu sehen, sind diese in folgender Tabelle \eqref{tb:evaQuartale} aufgezeigt. Die Beträge sind in Euro zu betrachten.

\bigskip
\begin{table}[!h]
    \begin{tabular}{lll}
        \cline{1-3} \rowcolor{gray}
                             & 1. Quartal & 2. Quartal \\ \cline{1-3} \rowcolor{white}
        Investiertes Kapital & 100000     & 200000     \\ \cline{1-3} \rowcolor{white}
        WACC                 & 8\%        & 8\%        \\ \cline{1-3} \rowcolor{white}
        NOPAT                & 10000      & 15000      \\ \cline{1-3} \rowcolor{white}
    \end{tabular}
    \caption{Übersicht der zwei Quartale von Unternehmen Muster}
    \label{tb:evaQuartale}
\end{table}

\subsection{Rechnung}

\subsubsection{Subtraktiver Ansatz}

Für das erste Quartal ergibt sich ein \ac{gwb} nach der Formel \eqref{eq:subtraktiver-geschaeftswertbeitrag} von:

\bigskip
$GWB_{Quartal 1} = 10000 \text{€} - 0,08 \cdot 100000 \text{€} = 2000 \text{€}$

\bigskip
\noindent
und für das zweite Quartal ein \ac{gwb} von:

\bigskip
$GWB_{Quartal 2} = 15000 \text{€} - 0,08 \cdot 200000 \text{€} = -1000 \text{€}$

\subsubsection{Multiplikativer Ansatz}

Unter Anwendung des multiplikativem Ansatzes \eqref{eq:mupliplikativ-gwb-geschaeftswertbeitrag} beläuft sich der \ac{gwb} ebenfalls für Quartal 1 auf 2000 € und für Quartal 2 auf -1000 €.

\bigskip
$GWB_{Quartal 1} = ((\frac{10000 \text{€}}{100000 \text{€}}) - 0,08) \cdot 100000 \text{€} = 2000 \text{€}$

\bigskip
$GWB_{Quartal 2} = ((\frac{15000 \text{€}}{200000 \text{€}}) - 0,08) \cdot 200000 \text{€} = -1000 \text{€}$


\subsubsection{Ergebnis}

In folgender Tabelle \eqref{tb:evaQuartaleBerechnet} sind die Ergebnisse nochmals übersichtlich dargestellt.

\begin{table}[!h]
    \begin{tabular}{lll}
        \cline{1-3} \rowcolor{gray}
                             & 1. Quartal & 2. Quartal \\ \cline{1-3} \rowcolor{white}
        Investiertes Kapital & 100000     & 200000     \\ \cline{1-3} \rowcolor{white}
        WACC                 & 8\%        & 8\%        \\ \cline{1-3} \rowcolor{white}
        NOPAT                & 10000      & 15000      \\ \hline \rowcolor{white}
        GWB                  & 2000       & -1000      \\ \cline{1-3} \rowcolor{white}
    \end{tabular}
    \caption{Übersicht der zwei Quartale von Unternehmen Muster inklusive \ac{gwb}}
    \label{tb:evaQuartaleBerechnet}
\end{table}

\subsection{Interpretation}

Nach dem das Unternehmen Muster im ersten Quartal ihren Wert um 2000 € steigern konnte, verlor sie im zweiten Quartal 1000 € an Wert. Im Gesamten konnte das Unternehmen zwar ihren Wert steigern, jedoch ist der Werteverlust ein Indiz dafür, dass es eine Fehleinschätzung gab. Diese könnte mögliche Investoren, die das Unternehmen Muster gerne bei ihren Projekten unterstützten würden, davon abhalten, da diese davon ausgehen könnten, dass das Unternehmen in Zukunft weiter an Werten verlieren wird.\\
Um im nächsten Quartal wieder eine Wertsteigerung zu erlangen, hat das Unternehmen Muster mehrere Möglichkeiten. Zum einen kann es das investierte Kapital senken. Hätte es im zweiten Quartal nur 150000 € anstatt den 200000 € investiert, hätte es eine Wertsteigerung von $15000 \text{€} - (0,08 \cdot 150000 \text{€}) =$ 3000 € gegeben. Des weiteren hätte das Unternehmen ein anderes Projekt mit einer höheren Rendite auswählen können. Hätte es ein Projekt mit einem \ac{wacc} von 7,5\% oder weniger investiert, wäre der\ac{gwb} neutral ($15000 \text{€} - (0,075 \cdot 200000 \text{€}) = 0 \text{€}$) oder positiv ($15000 \text{€} - (0,07 \cdot 200000 \text{€}) = 1000 \text{€}$). Zuletzt hätte das Unternehmen die Möglichkeit mehr Gewinn nach Steuern zu erwirtschaften. Wenn es dem unternehmen gelingen würde, 1000 € mehr, also einen \ac{nopat} von 16000 € zu erwirtschaften, wären sie neutral geblieben ($16000 \text{€} - (0,08 \cdot 200000 \text{€}) = 0 \text{€}$). Bei einem Gewinn von über 2000€ wäre der Wert des Unternehmens sogar gestiegen ($17000 \text{€} - (0,075 \cdot 200000 \text{€}) = 1000 \text{€}$).\footnote{\cite{studyflix-eva}} \footnote{\cite{controlling-eva}}

\section{Bewertung}

Der Geschäftswertbeitrag ist eine einfache Methode, um herauszufinden, ob das Unternehmen in der betrachteten Investitionsperiode Werte geschaffen oder vernichtet hat. Die Berechnung des \ac{gwb}s findet nur innerhalb einer Periode statt und bezieht sich meist auf vergangenheitsorientierten Werten. Deshalb ist eine Einschätzung über die Entwicklung in der Zukunft nicht ersichtlich. Der Freiheitsgrad eines Unternehmens, Anpassungen vorzunehmen, verringert die Vergleichbarkeit unterschiedlicher Jahre. Jedoch lässt sich durch den \ac{gwb} ein möglicher Strategiewechsel diskutieren.\footnote{\cite{controlling-eva}}
