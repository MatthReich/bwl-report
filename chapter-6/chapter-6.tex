\chapter{Geschäftswertbeitrag}
\label{Geschaeftswertbeitrag}

\section{Definition}

Der \ac{gwb}, im Englischen \ac{eva}, ist eine absolute Wertbeitragskennzahl, welche, vereinfacht, der Differenz zwischen den Kapitalerlösen und den Kapitalkosten entspricht. Demnach wird genau dann zusätzlicher Wert geschaffen, wenn über die Kapitalkosten für Eigen- und Fremdkapital hinaus verdient wird. Der \ac{gwb} wurde in den 1990er Jahren in der Unternehmensberatung Stern Stewart entwickelt.\footnote{\cite{wikipedia-eva}} \footnote{\cite{controlling-eva}}

\section{Berechnung}

Der Geschäftswertbeitrag setzt sich aus drei Elementen zusammen: Dem operativen Gewinn nach Steuern (\ac{nopat}), dem betriebsnotwendigem Vermögen (\ac{noa}) und den gewichteten durchschnittlichen Kapitalkosten (\ac{wacc}).\\

Der NOPAT ist der Teil, der Operativ entschieden wird. Hierbei geht es darum, dass man das Richtige machen möchte, beziehungsweise etwas besser machen. Die NOA bezieht sich auf eine Entscheidung basierend auf der Investition. Hierbei wird die Verbindlichkeit aus dem laufenden Geschäft nicht berücksichtig, ebenfalls das Ergebnis der Finanzierungstätigkeit. Zuletzt gibt es noch die Finanzierungsentscheidung, welches durch das WACC repräsentiert wird. Zudem kann das WAAC eine gesicherte Aussage über das Unternehmensrisiko geben.\footnote{\cite{bwllexicon-eva}} \footnote{\cite{wikipedia-eva}}\\ %% TODO rework

\noindent
Es gibt zwei Methoden, um den \ac{gwb} zu berechnen. Den subtraktiven Ansatz und den multiplikativen Ansatz. Beide Ansätze führen zum gleichen Berechnungsergebnis. Sie unterscheiden sich letztlich nur in der Fokussierung auf das absolute oder relative Erfolgsziel.

\subsection{Subtraktiver Ansatz}

Bei dem subtraktiven Ansatz werden von dem operativen Jahresergebnis die durchschnittlichen Kapitalkosten mal dem betriebsnotwendigem Vermögen abgezogen. Folgende Formel \eqref{eq:subtraktiver-geschaeftswertbeitrag}\footnote{\cite{wikipedia-eva}} repräsentiert diese Rechnung:

\begin{equation}
    GWB = NOPAT - WACC \cdot NOA
    \label{eq:subtraktiver-geschaeftswertbeitrag}
\end{equation}

\subsection{Multiplikativer Ansatz}

Bei dem multiplikativen Ansatz werden von der (Ist-)Gesamtkapitalrendite (\ac{irr}) die durchschnittlichen Kapitalkosten abgezogen und auf dieses Ergebnis wird dann das betriebsnotwendige Vermögen multipliziert. Dies wird in folgender Formel \eqref{eq:mupliplikativ-gwb-geschaeftswertbeitrag}\footnote{\cite{wikipedia-eva}} dargestellt. In der Formel \eqref{eq:mupliplikativ-irr-geschaeftswertbeitrag}\footnote{\cite{controllingportal-eva}} wird die Berechnung der \ac{irr} für die Vollständigkeit dargestellt. Die \ac{irr} berechnet sich aus dem Quotienten des operativen Gewinns nach Steuern und dem betriebsnotwendigem Vermögen multipliziert mit 100.

\begin{equation}
    GWB = (IRR - WACC) \cdot NOA
    \label{eq:mupliplikativ-gwb-geschaeftswertbeitrag}
\end{equation}

\begin{equation}
    IRR = \frac{NOPAT}{NOA} \cdot 100
    \label{eq:mupliplikativ-irr-geschaeftswertbeitrag}
\end{equation}

\bigskip

\noindent
Voraussetzung für diese Methode ist, dass der \ac{nopat} immer größer als die Kapitalkosten ist, welche bei der Investition anfallen, ist.\footnote{\cite{bwllexicon-eva}}

\section{Beispielrechnung}

Da der Fokus auf der Berechnung des Geschäftswertbeitrags liegt, werden für die Beispielrechnungen die Werte bereits angenommen. Demzufolge beträgt der \ac{wacc} 8\%, der \ac{nopat} beträgt 10000 € und die NOA belaufen sich auf 90000 €.

\subsection{Rechnung}

\subsubsection{Subtraktiver Ansatz}

Mit der Anwendung des subtraktiven Ansatzes \eqref{eq:subtraktiver-geschaeftswertbeitrag} ergibt sich folgende Rechnung:

\bigskip
GWB = 10000 € - 0,08 $\cdot$ 90000 € = 2800 €


\subsubsection{Multiplikativer Ansatz}

Da beide Ansätze dasselbe Ergebnis haben sollten, sollte auch der multiplikative Ansatz \eqref{eq:mupliplikativ-gwb-geschaeftswertbeitrag} einen GWB von 2800 ergeben:

\bigskip
GWB = (($\frac{10000}{90000})$ - 0,08) $\cdot$ 90000 € = 2800 €

\subsection{Interpretation}

Der Geschäftswertbeitrag von 2800 € zeigt, dass die Rendite über den Kosten für das eingesetzte Kapital liegt, weshalb diese Investition durchaus durchführbar ist. Wäre der Wert negativ, würde die Investition Verluste aufweisen und es wäre davon abzuraten, diese zu tätigen.

\section{Bewertung}

Der Geschäftswertbeitrag ist eine einfache Methode, um herauszufinden, ob das Unternehmen in der betrachteten Investitionsperiode Werte geschaffen oder vernichtet hat. Die Berechnung des \ac{gwb}s findet nur innerhalb einer Periode statt und bezieht sich meist auf vergangenheitsorientierten Werten. Deshalb ist eine Einschätzung über die Entwicklung in der Zukunft nicht ersichtlich. Der Freiheitsgrad eines Unternehmens, Anpassungen vorzunehmen, verringert die Vergleichbarkeit unterschiedlicher Jahre. Jedoch lässt sich durch den \ac{gwb} ein möglicher Strategiewechsel diskutieren.\footnote{\cite{controlling-eva}}
