\chapter{Dynamische Amortisationsrechnung}
\label{Dynamische Amortisationsrechnung}

\section{Definition}
Die dynamische Amortisationsrechnung ist ein Verfahren der dynamischen Investitionsrechnung. Sie wird auch öfters als Pay-off-Methode bezeichnet. Wie der Name es schon vermuten lässt, ist das Ziel des Verfahrens den Amortisationszeitpunkt einer Investition zu berechnen. Der Amortisationszeitpunkt beschreibt, den Punkt, an dem für die Investition verwendete Kapital wieder vollständig zurückgeflossen ist. Sie wird wie die anderen dynamischen Investitionsrechnungen verwenden, um das Risiko einer Investition zu bewerten oder um zwischen mehreren Investitionen die profitabelste auszuwählen.
\section{Unterschied zwischen statisch und dynamisch}
Der große Unterschied bei der statischen und dynamischen Amortisationsrechnung ist, dass bei der statischen Variante die Verzinsung des eingesetzten Kapitals unberücksichtigt bleibt. Dies hat zur Folge, dass bei der statischen Variante häufig eine günstigere Amortisationszeit berechnet wird. Diese statische Amortisationsdauer kann zu Entscheidungen führen, die auf ungenauen und falschen Ergebnissen beruhen. In der Praxis wird vor allem die dynamische Variante bevorzugt, weil sie Einzahlungen, Auszahlungen und Nutzungsperioden berücksichtigt. Zusätzlich werden in der dynamischen Amortisationsrechnung finanzmathematische Methoden verwendet, wie die Kapitalwertmethode.\footnote{\cite{gevestor}}
\newpage
\section{Interpretation}
Die Interpretation der dynamischen Amortisationsrechnung ist ähnlich wie bei der statischen Variante. Es wird der Amortisationzeitpunkt
berechnet, der den Rückfluss des investierten Kapitals einschließlich der Abzinsung auf die Mittel darstellt. Daraus folgt, dass die Amortisationszeit die Mindestnutzungsdauer eines Investitionsobjektes abbildet. Eine kürzere Amortisationszeit entspricht einem kleineren Investitionsrisiko. Genau wie bei den anderen Investitionsrechnungsverfahren, dient die Amortisationszeit als eine weitere Entscheidungshilfe bei der Bewertung von Investitionen.
\footnote{\cite{gevestor}}
\section{Beispiel}
Herr Mustermann will seine Produktion von Schuhen erhöhen. Dafür überlegt er sich in eine weitere Produktionsmaschine zu investieren. Die Maschine kostet 120 000 Euro und ist 4 Jahre nutzbar. Der kalkulierte Zinssatz beträgt außerdem 12 \%. Um eine Entscheidung zu treffen, führt er folgende Rechnung aus:

\begin{table}[!h]
    \begin{tabular}{lllllll}
        \cline{1-6} \rowcolor{gray}
        Jahr & Einzahlung & Auszahlung & Abzinsfaktor & Barwert  & Kapitalwert \\ \cline{1-6} \rowcolor{white}
        0    & 0          & 120.000    & $1,12^0$     & -120.000 & -120.000    \\ \cline{1-6} \rowcolor{white}
        1    & 69.000     & 30.000     & $1,12^1$     & 34821,43 & -85178,57   \\ \cline{1-6} \rowcolor{white}
        2    & 72.000     & 29.000     & $1,12^2$     & 35076,54 & -50102,04   \\ \cline{1-6} \rowcolor{white}
        3    & 68.000     & 31.000     & $1,12^3$     & 26335,87 & -23766,17   \\ \cline{1-6} \rowcolor{white}
        4    & 77.000     & 25.000     & $1,12^4$     & 33046,94 & 9280,77     \\ \cline{1-6} \rowcolor{white}
        4    & 20.000     & 0          & $1,12^4$     & 12710,36 & 21991,13    \\ \cline{1-6} \rowcolor{white}
    \end{tabular}
    \caption{Beispiel einer dynamischen Amortisationsrechnung auf 4 Jahre mit einem kalkulierten Zinssatz von 12\%}\label{tb:dynamische Amortisationsrechnung}
\end{table}
\bigskip
\noindent
Aus der Tabelle wird ersichtlich, dass sich der Amortisationszeitpunkt im vierten Jahr befindet, weil der Kapitalwert einen Wert über null annimmt. Daraus schlussfolgert Herr Mustermann, dass er die neue Maschine für die vier Jahre Anwendungszeit verwenden muss, um einen Gewinn zu produzieren. Weil die Maschine erst in Ihrem letzten Anwendungsjahr profitabel wird, entscheidet sich Herr Mustermann gegen die Investition.\footnote{\cite{payoff}}
\subsubsection{Erklärung}
Hier eine kurze Erklärung zur Tabelle. Wir rechnen zuerst, die Einzahlungen minus die Auszahlungen und dividieren diese durch den Abzinsfaktor. Das daraus resultierende Ergebnis wird auf den Kapitalwert drauf addiert und für jede Periode wiederholt. Wie man wahrscheinlich schon bemerkt hat, basiert diese Tabelle auf der Kapitalwertmethode. Der Unterschied ist jedoch, dass wir uns hier jede Periode einzeln anschauen, um den Amortisationszeitpunkt zu finden.
